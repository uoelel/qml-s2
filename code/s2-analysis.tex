% Options for packages loaded elsewhere
\PassOptionsToPackage{unicode}{hyperref}
\PassOptionsToPackage{hyphens}{url}
%
\documentclass[
]{article}
\usepackage{amsmath,amssymb}
\usepackage{lmodern}
\usepackage{iftex}
\ifPDFTeX
  \usepackage[T1]{fontenc}
  \usepackage[utf8]{inputenc}
  \usepackage{textcomp} % provide euro and other symbols
\else % if luatex or xetex
       %%% MODIFIED: unicode-math conflics with expex; mathspects conflicts with glossaries/leipzig
  \usepackage{fontspec} % \usepackage{unicode-math}
  \defaultfontfeatures{Scale=MatchLowercase}
  \defaultfontfeatures[\rmfamily]{Ligatures=TeX,Scale=1}
  \setmainfont[]{Arial}
\fi
% Use upquote if available, for straight quotes in verbatim environments
\IfFileExists{upquote.sty}{\usepackage{upquote}}{}
\IfFileExists{microtype.sty}{% use microtype if available
  \usepackage[]{microtype}
  \UseMicrotypeSet[protrusion]{basicmath} % disable protrusion for tt fonts
}{}
\makeatletter
\@ifundefined{KOMAClassName}{% if non-KOMA class
  \IfFileExists{parskip.sty}{%
    \usepackage{parskip}
  }{% else
    \setlength{\parindent}{0pt}
    \setlength{\parskip}{6pt plus 2pt minus 1pt}}
}{% if KOMA class
  \KOMAoptions{parskip=half}}
\makeatother
\usepackage{xcolor}
\IfFileExists{xurl.sty}{\usepackage{xurl}}{} % add URL line breaks if available
\IfFileExists{bookmark.sty}{\usepackage{bookmark}}{\usepackage{hyperref}}
\hypersetup{
  pdftitle={QML -- Summative Assessment 2},
  pdfauthor={YOUR EXAM NUMBER},
  hidelinks,
  pdfcreator={LaTeX via pandoc}}
\urlstyle{same} % disable monospaced font for URLs
\usepackage[margin=1in]{geometry}
\usepackage{graphicx}
\makeatletter
\def\maxwidth{\ifdim\Gin@nat@width>\linewidth\linewidth\else\Gin@nat@width\fi}
\def\maxheight{\ifdim\Gin@nat@height>\textheight\textheight\else\Gin@nat@height\fi}
\makeatother
% Scale images if necessary, so that they will not overflow the page
% margins by default, and it is still possible to overwrite the defaults
% using explicit options in \includegraphics[width, height, ...]{}
\setkeys{Gin}{width=\maxwidth,height=\maxheight,keepaspectratio}
% Set default figure placement to htbp
\makeatletter
\def\fps@figure{htbp}
\makeatother
\setlength{\emergencystretch}{3em} % prevent overfull lines
\providecommand{\tightlist}{%
  \setlength{\itemsep}{0pt}\setlength{\parskip}{0pt}}
\setcounter{secnumdepth}{5}
\ifLuaTeX
  \usepackage{selnolig}  % disable illegal ligatures
\fi

\title{QML -- Summative Assessment 2}
\author{YOUR EXAM NUMBER}
\date{2023-11-23}

\begin{document}
\maketitle

{
\setcounter{tocdepth}{2}
\tableofcontents
}
\section{Overview}\label{overview}

\textbf{PLEASE READ CAREFULLY}

\textbf{DUE Week 12 - Thu 07 December at noon}

\begin{itemize}
\tightlist
\item
  You must include your \textbf{exam number as the author} in the
  document preamble above.
\item
  Make sure you can render the Rmd file to PDF frequently while
  completing the assessment to avoid last minute panic!
\item
  Be careful with formatting. Make sure there is an empty line after
  each paragraph and/or code chunk.
\end{itemize}

\section{Read the data}\label{read-the-data}

\textbf{Instructions}:

\begin{itemize}
\tightlist
\item
  Read the chosen data.
\item
  Write below a brief description about the data:

  \begin{itemize}
  \tightlist
  \item
    Include the number of observations and the number of columns.
  \item
    Identify the variables/columns in the data that will be relevant to
    answer the research question, their type (categorical or numeric
    continuous/discrete) and, if categorical, report the levels of the
    variable.
  \item
    Report counts and/or summary measures of the relevant variables plus
    any other you think might be important to mention (for example, info
    about participants).
  \end{itemize}
\end{itemize}

\section{Plot the data}\label{plot-the-data}

\textbf{Instructions}:

\begin{itemize}
\tightlist
\item
  Now plot the data as you see fit. Remember the research question of
  the chosen data to help you decide which plots to include.
\item
  For each plot, write a brief description as text (include information
  on the components of the plot like axes, colour, panels etc and
  briefly describe the patterns you see).
\item
  Plotting the data is a good way of familiarising yourself with it, so
  that when it comes to modelling you have a better sense of what the
  data looks like!
\item
  \textbf{Include at least 3 plots}.
\end{itemize}

\section{Model the data}\label{model-the-data}

\textbf{Instructions}

\begin{itemize}
\tightlist
\item
  Finally, model the data and write a model report.
\item
  Pick a model formula that will help you answer the research question.
\item
  Follow these steps to help you model the data:

  \begin{itemize}
  \tightlist
  \item
    Identify the outcome variable and the predictors.
  \item
    Do you need to transform the variables and/or reorder levels?
  \item
    Identify the appropriate distribution for the outcome variable.
  \item
    Is an interaction term necessary?
  \item
    Do you need to centre any predictor?
  \end{itemize}
\end{itemize}

\section{Results}\label{results}

\textbf{Instructions}

\begin{itemize}
\tightlist
\item
  You should include at least two paragraphs.

  \begin{enumerate}
  \def\labelenumi{\arabic{enumi}.}
  \tightlist
  \item
    One reporting the model specifications (outcome, distribution of the
    outcome, predictors, coding, \ldots)
  \item
    One reporting the results from the model. To simplify things, here
    is some guidance. \textbf{Include at least the following}:

    \begin{itemize}
    \tightlist
    \item
      For each model coefficient (the ``population-level effects'' that
      get printed with \texttt{summary()}), report the 95\% CrI and the
      mean (\texttt{Estimate}) with the estimate error (i.e.~the SD,
      \texttt{Est.error}) of the posterior probability of the
      coefficient.
    \item
      For each combination of levels of categorical predictors, report
      the 95\% CrI of the conditional posterior probability.
    \item
      If your model includes a numeric predictor, you don't need to
      report conditional posterior probabilities depending on that
      predictor, but you can just report the 95\% CrI of the conditional
      posterior probabilities depending on levels of the categorical
      predictor, \emph{when the continuous predictor is at its mean}.
    \end{itemize}
  \item
    \textbf{Optional}:

    \begin{itemize}
    \tightlist
    \item
      If you wish, you can report the 95\% CrI of relevant differences
      between levels/values of categorical/numeric predictors.
    \end{itemize}
  \end{enumerate}
\item
  Include plots of the posterior distributions of each coefficient and
  the conditional posterior distributions.
\end{itemize}

\section{Discuss the results}\label{discuss-the-results}

\textbf{Instructions}:

\begin{itemize}
\tightlist
\item
  Based on the results that you reported above, try and answer the
  research question.
\item
  To give you an example of what we are looking for, here is a short
  paragraph on interpreting the results from the Week 8 lecture on
  morphological parsing.
\end{itemize}

\begin{quote}
\emph{As suggested by the 95\% CrI of the interaction term (in log-odds
{[}-1.72, 0.42{]}), there is quite a lot of uncertainty as to whether
the difference in probability of correct response in unrelated vs
constituent in right-branching pairs differs from that in left-branching
pairs, since the interval covers both negative and positive values.
Moreover, the conditional posterior probabilities of unrelated and
right-branching on the one hand and constituent and right branching on
the other are very similar, as can be seen in the plot above (and as
suggested by the respective 95\% CrIs: 90-97\% vs 91-97\%
respectively).}
\end{quote}

\end{document}
